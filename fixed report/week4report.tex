\documentclass[11pt]{article}

\usepackage{times}
\usepackage[english]{babel}

% -----------------------------------------------
% especially use this for you code
% -----------------------------------------------

\usepackage{courier}
\usepackage{listings}
\usepackage{color}
\usepackage{tabularx}
\usepackage{graphicx}

\definecolor{Gray}{gray}{0.95}

\definecolor{mygreen}{rgb}{0,0.6,0}
\definecolor{mygray}{rgb}{0.5,0.5,0.5}
\definecolor{mymauve}{rgb}{0.58,0,0.82}

\lstset{language=C++,
	basicstyle = \normalsize\ttfamily,   % the size and fonts that are used
	tabsize = 2,                    % sets default tabsize
	breaklines = true,              % sets automatic line breaking
	keywordstyle=\color{blue}\ttfamily,
	stringstyle=\color{red}\ttfamily,
	commentstyle=\color{mygreen}\ttfamily,
	numbers=left,
	keepspaces=true,
	showspaces=false,
	showstringspaces=false,
}

\begin{document}

\title{Programming in C/C++ \\
       Exercises set four: lexical scanners
}
\date{\today}
\author{Christiaan Steenkist \\
Jaime Betancor Valado \\
Remco Bos \\
}

\maketitle
\section*{Exercise 23, Embedded patterns}
We were tasked to construct a program that uses a scanner for printing all the sorted words finded in a piece of text.
\subsection*{Code listings}
\lstinputlisting[caption = main.ih]{src/a23/main.ih}
\lstinputlisting[caption = main.cc]{src/a23/main.cc}
\lstinputlisting[caption = lexer.ll]{src/a23/lexer.ll}
\lstinputlisting[caption = Scanner.h]{src/a23/Scanner.h}
\section*{Exercise 24, Non-greedy matching}
We made a lexical scanners that performs non-greedy matching.

\subsection*{Code listings}
\lstinputlisting[caption = main.ih]{src/a24/main.ih}
\lstinputlisting[caption = main.cc]{src/a24/main.cc}
\lstinputlisting[caption = lexer.ll]{src/a24/lexer.ll}
\lstinputlisting[caption = Scanner.h]{src/a24/Scanner.h}

\section*{Exercise 26}
See exercise 28.

\section*{Exercise 27, tokens}
Why are there so many operators?

\subsection*{Lexer}
\lstinputlisting[caption = lexer.ll]{src/a27/lexer.ll}

\section*{Exercise 28, scanner that scans text}
We were asked to design a scanner thats scan a piece of text.

\subsection*{Code listings}
\lstinputlisting[caption = main.h]{src/a26/main.h}
\lstinputlisting[caption = main.ih]{src/a26/main.ih}
\lstinputlisting[caption = main.cc]{src/a26/main.cc}
\lstinputlisting[caption = getrawchar.cc]{src/a26/getrawchar.cc}
\lstinputlisting[caption = dequote.cc]{src/a26/dequote.cc}
\lstinputlisting[caption = makeraw.cc]{src/a26/makeraw.cc}

\subsubsection*{Lexer and scanner}
\lstinputlisting[caption = lexer.ll]{src/a26/lexer.ll}
\lstinputlisting[caption = Scanner.h]{src/a26/Scanner.h}

\subsubsection*{Writer}
\lstinputlisting[caption = writer.addString.cc]{src/a26/writer.addString.cc}
\lstinputlisting[caption = writer.constr.cc]{src/a26/writer.constr.cc}
\lstinputlisting[caption = writer.grab.cc]{src/a26/writer.destr.cc}
\lstinputlisting[caption = writer.writecode.cc]{src/a26/writer.writecode.cc}

\subsubsection*{Input}
\lstinputlisting[caption = test]{src/a26/test.txt}

\subsubsection*{Output}
\lstinputlisting[caption = test.cc]{src/a26/testfile/test.cc}
\lstinputlisting[caption = test.gsl]{src/a26/testfile/test.gsl}

\end{document}